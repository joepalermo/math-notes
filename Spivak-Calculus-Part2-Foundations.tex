\documentclass{article}
\usepackage{amsfonts}
\begin{document}
\title{%
  Calculus, Part 2: Foundations \\
  \large Chapter Summaries}
\author{}
\date{}
\maketitle

\section*{Chapter 3: Functions}

\begin{flushleft}
The chapter begins with a provisional definition of a function as: a rule which assigns, to certain real numbers, some other real number. A function is any rule, not just one that can be expressed by an algebraic formula...nor is it necessarily a rule can be applied in practice."
\end{flushleft}

\begin{flushleft}
The set of numbers to which a function applies is the \textbf{domain}.
\end{flushleft}

\begin{flushleft}
Polynomial functions are introduced, and the degree is defined as the nonzero coefficient of the highest power.
\end{flushleft}

\begin{flushleft}
Rational functions are defined as quotients of polynomial functions.
\end{flushleft}

\begin{flushleft}
Elementary operations (+, -, *, /) are defined over functions. The domain of a function built by elementary operations from other functions is the intersection of the domains of all the functions. There is a caveat when '/' is used that for example if h = f/g, then g $\neq$ 0.
\end{flushleft}

\begin{flushleft}
Associativity, and commutativity over addition and multiplication of functions are shown to be easy to prove.
\end{flushleft}

\begin{flushleft}
Function composition is introduced. Note that composition is not generally commutative, although it is associative
\end{flushleft}

\begin{flushleft}
A more precise definition for a function is given: A \textbf{function} is a collection of pairs of numbers such that if $(a,b)$ and $(a,c)$ are both in the collection, then b=c.
\end{flushleft}

\begin{flushleft}
Similarly a more precise definition is given for domain: The domain of f is the set of all a for which there is some b such that $(a,b)$ is in f.
\end{flushleft}

\section*{Chapter 4: Graphs}

\begin{flushleft}
Spivak notes: "[the] method of 'drawing' numbers is intended solely as a method of picturing certain abstract ideas, and our proofs will never rely on these pictures."
\end{flushleft}

\begin{flushleft}
The set ${x: a < x < b}$ is denoted by $(a,b)$ and called the open interval from a to b. The set ${x: a \leq x \leq b}$ is denoted by $[a,b]$ and called the closed interval from a to b.
\end{flushleft}

\begin{flushleft}
Distance between points $(a,b)$ and $(c,d)$ is defined $\sqrt{(a-c)^2 + (b-d)^2}$.
\end{flushleft}

\begin{flushleft}
Functions of the form $f(x) = cx + d$ are called \textbf{linear functions}, and functions of the form $f(x) = x^n$ are called power functions.
\end{flushleft}

\begin{flushleft}
It is noted that a polynomial function of degree n will have at most n-1 local minima and maxima, though it may be much smaller. Spivak notes that "Although these assertions are easy to make, we will not even contemplate giving proofs until Part 3 (once the powerful methods of part 3 are available)."
\end{flushleft}

\begin{flushleft}
Various graphs are shown to demonstrate that some functions are impossible to accurately convey in a drawing.
\end{flushleft}

\begin{flushleft}
Some graphs are discussed which are not functions, namely the circle. A circle with center $(a,b)$ and radius $r > 0$ contains all points $(x,y)$ with $\sqrt{(x-a)^2 + (y-b)^2} = r$. Also discussed is the ellipse which is defined as the set of points the sum of whose distances from 2 foci is a constant, and the hyperbola is defined analogously, except that we require the difference of the 2 distances to be a constant. Though these are not functions, they can be reduced to the functions of which they are composed.
\end{flushleft}

\begin{flushleft}
Spivak concludes by noting: "A mathematical definition of this concept [the reasonableness of a function] is by no means easy, and a great deal of this book may be viewed as successive attempts to impose more and more conditions that a 'reasonable' function must sastisfy."
\end{flushleft}

\end{document}

\documentclass{article}
\begin{document}
\title{Calculus - Chapter Notes}
\author{}
\date{}
\maketitle

\section{Chapter 1: Basic Properties of Numbers}

\begin{enumerate}
	\item additive associativity
	\item additive identify
	\item additive inverse
	\item additive commutativity
	\item multiplicative associativity
	\item multiplicative identify
	\item multiplicative inverse
	\item multiplicative commutativity
	\item distributive law
	\item trichotomy law
	\item closure under addition for P
	\item closure under multiplication for P
\end{enumerate}

\begin{flushleft}
With P1-P8 very little can actually be proved. But with the introduction of P9 a lot more can be demonstrated. For instance, it can be shown that $(-a) \cdot (-b) = a \cdot b$. Furthermore P12 is actually the justification for almost all algebraic manipulations.
\end{flushleft}

\begin{flushleft}
Inequality symbols such as $<$ and $\leq$ and the absolute value are defined as one would expect in terms of P (introduced in P10-P12).
\end{flushleft}

\textbf{Theorem 1}: For all numbers a and b, we have $ |a+b| \leq |a|+|b| $

\begin{flushleft}
Some concluding notes from Spivak: "[though at this point in the book] we do not yet thoroughly understand numbers; we may still say that, in whatever way numbers are finally defined, they should certainly have properties P1-P12...It is still a crucial question whether P1-P12 actually account for textit{all} properties of numbers. As a matter of fact, we shall soon see that they do textit{not}. In the next chapter the deficiencies of P1-P12 will become quite clear, but the proper means for correcting these deficiencies is not so easily discovered [it will require all of part 2 of the book]"
\end{flushleft}

\section{Chapter 2: Numbers of Various Sorts}


\end{document}
